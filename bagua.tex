% !TEX TS-program = xelatex
% !TEX encoding = UTF-8

% This is a simple template for a XeLaTeX document using the "article" class,
% with the fontspec package to easily select fonts.

\documentclass[11pt]{article} % use larger type; default would be 10pt

\usepackage{fontspec} % Font selection for XeLaTeX; see fontspec.pdf for documentation
\defaultfontfeatures{Mapping=tex-text} % to support TeX conventions like ``---''
\usepackage{xunicode} % Unicode support for LaTeX character names (accents, European chars, etc)
\usepackage{xltxtra} % Extra customizations for XeLaTeX
\usepackage{tikz}

\setmainfont{SourceHanSerifCN-Regular.otf} % set the main body font (\textrm), assumes Charis SIL is installed
%\setsansfont{Deja Vu Sans}
%\setmonofont{Deja Vu Mono}

% other LaTeX packages.....
\usepackage{geometry} % See geometry.pdf to learn the layout options. There are lots.
\geometry{a4paper} % or letterpaper (US) or a5paper or....
%\usepackage[parfill]{parskip} % Activate to begin paragraphs with an empty line rather than an indent

\usepackage{graphicx} % support the \includegraphics command and options

\title{先天八卦图}
\author{The Author}
%\date{} % Activate to display a given date or no date (if empty),
         % otherwise the current date is printed 

\begin{document}
\maketitle

%\section{}

%\subsection{}


% 八四二一码 0 1 2 3 4 5 6 7
% 先天八卦码 2 3 7 6 5 4 0 1


\begin{tikzpicture}

\def\rad{6}
\def\bargap{0.7}
\def\barwidth{3}
\def\linewidth{10}
\def\barbreak{1}
\def\colorying{white}
\def\coloryang{black}


\def\w{\barwidth/2}
\def\r{\rad+\bargap*0}
\def\b{\barbreak/2}


\foreach \i/\b in {0,2,  1,3,  2,7,  3,6,  4,5,  5,4,  6,0,  7,1}
\draw [\coloryang] [rotate=\i*45] [line width=\linewidth] (\r,-\w) -- (\r,\w);

\foreach \i in {0, 1, 6, 7}
\draw [\colorying] [rotate=\i*45] [line width=\linewidth] (\r,-\b) -- (\r,\b);



\def\r{\rad+\bargap*1}
\foreach \i/\b in {0,2,  1,3,  2,7,  3,6,  4,5,  5,4,  6,0,  7,1}
\draw [\coloryang] [rotate=\i*45] [line width=\linewidth] (\r,-\w) -- (\r,\w);

\foreach \i in {4, 5, 6, 7}
\draw [\colorying] [rotate=\i*45] [line width=\linewidth] (\r,-\b) -- (\r,\b);



\def\r{\rad+\bargap*2}
\foreach \i/\b in {0,2,  1,3,  2,7,  3,6,  4,5,  5,4,  6,0,  7,1}
\draw [\coloryang] [rotate=\i*45] [line width=\linewidth] (\r,-\w) -- (\r,\w);

\foreach \i in {0, 3, 5, 6}
\draw [\colorying] [rotate=\i*45] [line width=\linewidth] (\r,-\b) -- (\r,\b);

% yingyang fish outline

%\draw[red]   (4, 0) arc(0:360:4);

%\draw[green] (0, 0) arc(-90:90:2);
%\draw[green] (0, 0) arc(90:270:2);

%\draw[blue]  (1, 2) arc(0:360:1);
%\draw[blue]  (1,-2) arc(0:360:1);

% yingyang fish
% these colors are for debbug purpose
\def\green{black}
\def\blue{black}
\def\yellow{white}
\def\orange{white}
\def\red{white}


\draw (4, 0) arc(0:360:4);
\fill[\green]  (0,-4) arc(-90:90:4);
\fill[\yellow] (0, 4) arc(90:270:4);

\fill[\blue]   (0, -2) circle(2);
\fill[\orange] (0, 2) circle(2);

\fill[\blue] (0, 2) circle(1);
\fill[\red]  (0,-2) circle(1);

%\node at (0, 5) {qian};

% sqrt(2) = 1.414
\def\r{5}
\def\rr{\r/1.414}


\node [rotate=6*45] at (\r, 0) {坎};
\node [rotate=7*45] at (\rr, \rr) {巽};

\node [rotate=0*45] at (0, \r) {乾};
\node [rotate=1*45] at (-\rr, \rr) {兑};


\node [rotate=2*45] at (-\r, 0) {离};
\node [rotate=3*45] at (-\rr, -\rr) {震};

\node [rotate=4*45] at (0, -\r) {坤};
\node [rotate=5*45] at (\rr, -\rr) {艮};



\end{tikzpicture}

\end{document}
