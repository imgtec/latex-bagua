% !TEX TS-program = xelatex
% !TEX encoding = UTF-8

% This is a simple template for a XeLaTeX document using the "article" class,
% with the fontspec package to easily select fonts.

\documentclass[11pt]{article} % use larger type; default would be 10pt

\usepackage{fontspec} % Font selection for XeLaTeX; see fontspec.pdf for documentation
\defaultfontfeatures{Mapping=tex-text} % to support TeX conventions like ``---''
\usepackage{xunicode} % Unicode support for LaTeX character names (accents, European chars, etc)
\usepackage{xltxtra} % Extra customizations for XeLaTeX
\usepackage{tikz}

\setmainfont{Courier New} % set the main body font (\textrm), assumes Charis SIL is installed
%\setsansfont{Deja Vu Sans}
%\setmonofont{Deja Vu Mono}

% other LaTeX packages.....
\usepackage{geometry} % See geometry.pdf to learn the layout options. There are lots.
\geometry{a4paper} % or letterpaper (US) or a5paper or....
%\usepackage[parfill]{parskip} % Activate to begin paragraphs with an empty line rather than an indent

\usepackage{graphicx} % support the \includegraphics command and options

\title{Brief Article}
\author{The Author}
%\date{} % Activate to display a given date or no date (if empty),
         % otherwise the current date is printed 

\begin{document}
\maketitle

%\section{}

%\subsection{}


% ���Ķ�һ�� 0 1 2 3 4 5 6 7
% ��������� 2 3 7 6 5 4 0 1


\begin{tikzpicture}
\draw [line width=12] (4,-1) -- (4,1);

\draw [rotate=45] [line width=12] (4,-1) -- (4,1);
\draw [rotate=90] [line width=12] (4,-1) -- (4,1);
\draw [rotate=135] [line width=12] (4,-1) -- (4,1);
\draw [rotate=180] [line width=12] (4,-1) -- (4,1);

\draw [rotate=-45] [line width=12] (4,-1) -- (4,1);
\draw [rotate=-90] [line width=12] (4,-1) -- (4,1);
\draw [rotate=-135] [line width=12] (4,-1) -- (4,1);
\draw [rotate=-180] [line width=12] (4,-1) -- (4,1);


\draw [line width=12] (4.5,-1) -- (4.5,1);

\draw [rotate=45] [line width=12] (4.5,-1) -- (4.5,1);
\draw [rotate=90] [line width=12] (4.5,-1) -- (4.5,1);
\draw [rotate=135] [line width=12] (4.5,-1) -- (4.5,1);
\draw [rotate=180] [line width=12] (4.5,-1) -- (4.5,1);

\draw [rotate=-45] [line width=12] (4.5,-1) -- (4.5,1);
\draw [rotate=-90] [line width=12] (4.5,-1) -- (4.5,1);
\draw [rotate=-135] [line width=12] (4.5,-1) -- (4.5,1);
\draw [rotate=-180] [line width=12] (4.5,-1) -- (4.5,1);


\draw [line width=12] (5,-1) -- (5,1);

\draw [rotate=45] [line width=12] (5,-1) -- (5,1);
\draw [rotate=90] [line width=12] (5,-1) -- (5,1);
\draw [rotate=135] [line width=12] (5,-1) -- (5,1);
\draw [rotate=180] [line width=12] (5,-1) -- (5,1);

\draw [rotate=-45] [line width=12] (5,-1) -- (5,1);
\draw [rotate=-90] [line width=12] (5,-1) -- (5,1);
\draw [rotate=-135] [line width=12] (5,-1) -- (5,1);
\draw [rotate=-180] [line width=12] (5,-1) -- (5,1);



\def\rad{6}
\def\gap{0.6}


\def\r{\rad+\gap}
\foreach \i/\b in {0,2,  1,3,  2,7,  3,6,  4,5,  5,4,  6,0,  7,1}
\draw [rotate=\i*45] [line width=12] (\r,-1) -- (\r,1);

\def\r{\rad+\gap+\gap}
\foreach \i/\b in {0,2,  1,3,  2,7,  3,6,  4,5,  5,4,  6,0,  7,1}
\draw [rotate=\i*45] [line width=12] (\r,-1) -- (\r,1);

\def\r{\rad+\gap+\gap+\gap}
\foreach \i/\b in {0,2,  1,3,  2,7,  3,6,  4,5,  5,4,  6,0,  7,1}
\draw [rotate=\i*45] [line width=12] (\r,-1) -- (\r,1);



\end{tikzpicture}

\end{document}
